\chapter{Fundamentação Teórica}
\label{ch:identificador}

Baseado no artigo \cite{yanique2023drone} que realiza uma analise SWOT que é uma ferramenta usada para avaliar os pontos fortes, fracos, oportunidades e ameaças de um determinado negócio ou tecnologia.
\begin{itemize}
    \item \textbf{Pontos fortes:} Acessibilidade melhorada – os drones podem fazer entregas em áreas remotas e de difícil acesso, proporcionando melhor acessibilidade para clientes e empresas.
    \item \textbf{Fraquezas:} Capacidade de carga útil limitada – os drones têm capacidade de carga útil limitada, o que significa que podem não ser adequados para pacotes maiores e mais pesados.
    \item \textbf{Oportunidade:} Aumento da demanda – com o tempo houve uma demanda muito grande por métodos de entrega mais rápidos e eficientes, e com isso a entrega feita por drones é uma solução muito atraente.
    \item \textbf{Ameaças:} Percepção publica – a percepção publica sobre os drones pode ser uma ameaça potencial para a indústria, em especifico se os drones forem visto como um incomodo ou ameaça a segurança pública.
\end{itemize}

    \section{Vantagens, economia e desafios}

Vantagens e Potencial Econômico
\begin{itemize}
    \item O uso de drones para entrega de pacotes oferece várias vantagens em relação aos métodos tradicionais. Eles proporcionam maior flexibilidade, acessibilidade e eficiência na entrega, além de benefícios ambientais, como a redução de emissões de CO2 quando comparados a veículos movidos a gasolina \cite{cornell2023drones}.
    \item As vantagens econômicas e ambientais sugerem que a entrega por drones pode se tornar uma parte importante do ecossistema de entregas. Empresas com visão de futuro planejarão hoje um futuro habilitado para drone \cite{cornell2023drones}.
\end{itemize}

    \section{Desafios tecnológicos e regulatórios}
    
Embora promissores, os drones enfrentam vários desafios. Os custos
operacionais atuais são altos, principalmente devido à necessidade de operadores humanos monitorando cada drone. Para que os drones se
tornem mais competitivos, será necessário avançar em tecnologias de voo autônomo e na gestão de tráfego aéreo, permitindo que um operador gerencie vários drones simultaneamente \cite{cornell2023drones}. Além disso, as regulamentações precisam evoluir para permitir uma maior densidade de drones no espaço aéreo urbano.

    \section{Melhora nas entregas}

No artigo \cite{Submit2022} faz a citação que os drones podem realizar entregas de forma significativamente mais rápida do que os métodos tradicionais. Foi realizado um teste em Belo Horizonte (MG), o tempo de delivery caiu de 40 minutos para 5 minutos.

	\section{Steve Robot Express}

Com Base em pesquisas, procuramos trabalhar com o objetivo de pesquisa em informações advinda de autores que pudessem fundamentar uma base de reflexões teóricas e práticas para contribuir na criação da nossa empresa focando nas novas modalidades de entregas autônomas e mais sustentáveis por robôs Autônomos.

De acordo com \cite{Bakach2020} a utilização de robôs poderia contribuir para a redução dos custos como das emissões de gases do efeito estuda (GEE) e redução de congestionamento em grandes cidades. Também devemos pensar nos desafios de se viabilizar todo esse processo de utilização dos robôs autônomos abordando os desafios de agendar entregas utilizando os robôs \cite{BoysenN2018}. Podemos nos basear em cálculos matemáticos que nos fornecem dados valiosos para o aprimoramento dos robôs a cada entrega realizada. 

Outra possibilidade seria em áreas de pilotagem de aeronaves por exemplos próximos de campos de voos e aeroportos \cite{faa2018} onde os robôs enfrentam menos regulamentações de segurança do que a de drones.

Empresas como Starship Technologies, Robby, Amazon Scout e Synkar já estão vendendo ou desenvolvendo robôs de entrega buscando ganhar espaço em um mercado de robôs de serviço que prevê um aumento na demanda de entregas nos próximos anos com o envelhecimento da população e aumento na demanda por entregas. (Boston Consulting Group (BCG)) (Mordor Intelligence Research \& Advisory).
