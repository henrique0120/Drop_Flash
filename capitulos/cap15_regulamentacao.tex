\chapter{Regulamentação de robôs entregadores}
\label{ch:identificador}

No brasil as regulamentações dos robôs entregadores ainda estão em 
desenvolvimento, mas já existem algumas diretrizes gerais que podem ser  aplicadas com base nas regras existentes como a NR 12 Norma Reguladora \cite{NR12} criada em 1978 (lei 3.214), criada pelo governo federal com o objetivo de estabelecer os parâmetros técnicos e quais regras cumprir para oferecer proteção no trabalho com diferentes maquinários. Dentro dessa área temos também a ISO/TS 15066 \cite{ISO/TS15066} voltada para robôs colaborativos sendo um complemento da ISO 10218 \cite{(ISO10218)} que fala sobre “Requerimentos de Segurança para Robôs Industriais”.

Com base nessas normas podemos encontrar regras já existentes para dispositivos autônomos e veículos que devem ser considerados na implantação do nosso robô de entregas Steve sendo essas.

\textbf{Segurança:} Garantir que os robôs sejam projetados com dispositivos de segurança evitando acidentes com pedestres e automóveis. Permitindo que sejam capazes de evitar obstáculos e interagir de forma segura com o ambiente.

\textbf{Tráfego:} Implementação de regras especificas para circulação dos robôs nas vias de condomínios residenciais, como limites de velocidade, uso de faixas e sinalização adequada.

\textbf{Privacidade:} Garantir a proteção de dados e a privacidade dos cidadãos, o que inclui a gestão das informações coletadas pelos sensores dos robôs durante a entrega.

\textbf{Responsabilidade:} A Drop Flash junto a Synkar nos responsabilizamos por danos ou acidentes envolvendo robôs entregadores caso ocorram prestando auxílio em possíveis ocorrências. 

\textbf{Homologação:} Com base na NR 12 e ISO/TS 15066 buscamos junto aos órgãos competentes garantir que atendam aos padrões técnicos e de segurança exigidos.

\textbf{Operação Comercial:} Para obtenção de permissões para operações comerciais em diversas capitais entramos em contato com órgãos reguladores para obtenção de licenças ou autorizações dependendo das regras de cada estado

Nos Estados Unidos a Federal Aviation Administration \cite{faa2018} estabelecem menos regulamentações de segurança em áreas de pilotagem de aeronaves para robôs autônomos facilitando assim a implantação dos robôs em aeroportos ou locais próximos a áreas de carga e descarga de aeronaves. 