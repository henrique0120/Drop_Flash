\chapter{Conclusões}
\label{ch:conclusao}

A operação da Drop Flash, uma empresa de entregas que utiliza tecnologias avançadas, incluindo robôs autônomos, como o Steve Robot Express, e drones de diferentes capacidades. Inicialmente, detalhou-se a operacionalização da Drop Flash conectando clientes a entregadores independentes e estabelecendo um sistema de taxas baseado na distância percorrida e na demanda regional. Especificamente, a implementação do Steve Robot Express em condomínios foi minuciosamente analisada, considerando fatores como o número de residências, pedidos mensais, distância percorrida e tempo de entrega. Calculou-se a necessidade de robôs com base na autonomia e capacidade operacional dos Steve Robots, sugerindo a instalação de duas unidades para assegurar eficiência e rapidez nas entregas. As características técnicas dos Steve Robots, incluindo suas múltiplas câmeras, sensores ultrassônicos e sistemas de navegação GPS, foram destacadas como fundamentais para a operação segura e eficiente em ambientes diversos. Além disso, o documento explorou os custos e o retorno sobre o investimento para a implementação dos Steve Robots. A análise financeira indicou um retorno atrativo de 30,75\% ao final de doze meses para um condomínio de 328 unidades, demonstrando a viabilidade econômica do projeto. A Drop Flash também investe em drones para otimizar entregas rápidas e de menor peso. Os drones DLV-1 e DLV-2 foram descritos com suas respectivas capacidades de carga e distâncias operacionais, destacando suas vantagens para entregas urgentes e de maior peso, respectivamente. Por fim, discutiu-se as vantagens do tráfego pago para a empresa, ressaltando a capacidade de gerar resultados imediatos, focar nas conversões e a flexibilidade de orçamento. O tráfego pago permite uma segmentação avançada e uma estratégia baseada em dados, essencial para o aprimoramento contínuo das campanhas publicitárias e maximização dos resultados. Conclui-se que a Drop Flash, ao integrar robôs autônomos e drones em suas operações, não apenas melhora a eficiência e rapidez das entregas, mas também se posiciona de forma competitiva no mercado emergente de robótica de serviços. A análise detalhada dos custos, retorno sobre o investimento e as estratégias de marketing reforçam a sustentabilidade e escalabilidade do negócio, prometendo um crescimento significativo nos próximos anos.