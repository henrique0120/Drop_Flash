% !TEX root = ..\main.tex

\PalavrasChave{Delivery, Eficiência, Robôs, Drones, Motos, Sustentabilidade.}
\centeredchapterstyle
\begin{resumo}
    \noindent\textbf{}A crescente demanda por serviços de delivery, especialmente em áreas urbanas densamente povoadas, tem gerado preocupações significativas em relação à eficiência e rapidez das entregas. Este estudo propõe soluções para mitigar os desafios associados a essa alta demanda, visando reduzir os tempos de espera e melhorar a experiência do cliente. O objetivo principal desta pesquisa é avaliar a eficácia de diferentes modalidades de entrega, incluindo o uso de robôs, drones e motos elétricas, na otimização da produtividade e na redução dos custos operacionais e ambientais. Para alcançar esses objetivos, foram realizadas investigações tanto individualmente quanto em grupo, além de múltiplas reuniões para discutir os temas pertinentes. Métodos de análise de dados qualitativos e quantitativos foram empregados para avaliar a viabilidade e eficácia das diferentes abordagens de entrega. Os resultados indicam que a implementação de métodos de entrega utilizando robôs, drones e motos elétricas é altamente viável, essas abordagens demonstraram capacidade para aumentar a eficiência das entregas, reduzir os custos operacionais e minimizar o impacto ambiental associado ao transporte de mercadorias. A adoção de tecnologias inovadoras, como robôs, drones e motos elétricas, pode representar uma solução eficaz para os desafios enfrentados no setor de delivery. Além de aumentar a satisfação do cliente através da redução dos tempos de espera, essas soluções também oferecem benefícios significativos para as empresas, incluindo a redução de custos e o aumento do retorno líquido. No entanto, são necessários estudos adicionais para avaliar a implementação prática e a aceitação do mercado dessas tecnologias em larga escala.
\end{resumo}